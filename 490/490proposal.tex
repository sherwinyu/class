
\documentclass{article}


\begin{document}
\title{An Implementation of a SPDY Web Server in Go}
\author{Sherwin Yu}
% Advisor: Daniel Abadi}
\date{February 26th, 2012}
\maketitle

\section{Motivation}
\label{introduction}
The purpose of this project is to implement and benchmark a minimalistic SPDY web server using Go. Go was designed as a systems programming language and is relatively new; writing
a web server from scratch will be an excellent case study of its effectiveness as a systems programming language. SPDY is also a relatively new experimental protocol, and as of
writing, no open source Go implementation of a SPDY server exists. 

\subsection{Motivation}
\label{motivation}
SPDY is new network protocol developed at Google designed to reduce web page load latency. It is an application-layer protocol that complements HTTP by improving how HTTP
requests and responses are handled. SPDY prioritizes and multiplexes web page resources to minimize the number connections needed; by multiplexing concurrent streams, many
requests and responses are interleaved on the same channel, so greater network efficiency is achieved. SPDY supports different reequest priorities to prevent the client from
blocking requests on in limited bandwidth situations. The client can assign different priorities to each item it requests and the server will respond appropriately.  SPDY also features HTTP header compression, resulting in less data to
transfer.

SPDY was designed to minimize deployment complexity: because it is built ontop of TCP, modifications are needed only at the server and client endpoints. Some Google servers already support SPDY; SPDY compatible builds of Chromium and Chrome exist.

\subsection{The Go programming language}
Go is a systems programming language developed and supported by Google with native support for concurrency. Go was designed with expressiveness, conciseness, and code cleaness in
mind for writing applications on multicore and networked machines. Go is compiled, statically typed, and garbage collected, yet was designed to have the expressiveness of a
dynamically typed interpreted language. Go features Goroutines, lightweight functions executing with other
goroutines in the same address space. Goroutines are multiplexed onto multiple OS threads by the Go runtime, hiding many of the complexities of thread creation and management.


\section{Goals}
\label{goals}

\subsection{Implementation of a minimalistic SPDY server}
Implementation of the core features of SPDY will be supported, depending on how much time is available. Accordingly, a minimalistic testing client will be created as well. Testing
can also be done using a SPDY compliant browser Chromium or Chrome.

\subsection{Comparison of web server architectures}
If time allows, implementation of different web server architectures such as asynchronous, thread per request, and thread pool to handle concurrent requests. Doing so will provide
valuable experience on the actual use of Go's support for concurrency in Goroutines and channels. Benchmarking different servers will also provide insight to Go's performance;
comparison to other HTTP server implementations will also give a idea of Go's performance relative to other frameworks.

\subsection{Network benchmarking}
Comparison of HTTP vs SPDY server implementations and network analysis under varying situations will provide insight to SPDY as an actual protocol; though much of this analysis has
already been done, this can be an independent verification of Google's claims and also an analysis of the core features of SPDY. 

\section{Deliverables}
\label{deliverables}
These are the deliverables for this project.

\subsection{Code}

\label{labelStatement}
The code will be written in Go and will be made available and
open sourced via GitHub.

\subsection{Report}
\label{pres}
I will complete a report covering the background, problem, goals, methods, and
results.


% \begin{thebibliography}{99}
% \bibitem{dean} Jeffrey Dean and Sanjay Ghemawat. Mapreduce: simplified data
% processing on large clusters. Commun. ACM, 51:107-113, January 2008.
% \bibitem{fischer} Michael J. Fischer, Xueyuan Su, and Yitong Yin. Assigning
% tasks for efficiency in hadoop: extended abstract. In Proceedings of the 22nd
% ACM symposium on Parallelism in algorithms and architectures, SPAA '10, pages
% 30-39, New York, NY USA, 210. ACM
% \bibitem{ghemawat} Sanjay Ghemawat, Howard Gobioff, and Shun-Tak Leung. The
% google file system. SIGOPS Oper. Syst. Rev. 37:29-43, October 2003.
% \bibitem{Zaharia}. Matei Zaharia, Andrew Konwinski, Anthony D. Joseph, Randy H. Katz,
% and Ion Stoica. Improving mapreduce performance in heterogeneous envi-
% ronments. Technical Report UCB/EECS-2008-99, EECS Department, Uni-
% versity of California, Berkeley, Aug 2008.
%  \end{thebibliography}

% Stop your text
\end{document}

